% LaTeX
\documentclass[a4paper,10pt]{article}

\usepackage[utf8]{inputenc}
\usepackage[T1]{fontenc}
\usepackage[english]{babel}
\usepackage[left=1.5cm,right=1.5cm]{geometry}
\usepackage{graphicx}
\usepackage{booktabs}
\usepackage{amsmath}
\usepackage{siunitx}


\title{Notes on the extended grid}
\author{Joar Bagge}
\date{2020-10-13}


\begin{document}

\maketitle

\section{Baseline case}

Plots produced by \verb|../comp_fig5.m| and \verb|../comp_figT1.m|,
using 100 source particles in a cube of side length $L=1$. For
other parameters, see the scripts. In this baseline case, the
grid size $M$ is increased by $P$ (the number of points in the
window function's support) in the nonperiodic directions (this is
needed to fit the window function without clipping or wrapping),
and the side length is also increased to keep $h=L/M$ constant
across all three directions.

{\noindent\centering
\includegraphics[width=0.5\textwidth]{fig/x-f1-baseline}\\%
\includegraphics[width=0.5\textwidth]{fig/x-f2-baseline}%
\includegraphics[width=0.5\textwidth]{fig/x-f3-baseline}\\%
\includegraphics[width=0.5\textwidth]{fig/x-f4-baseline}%
\includegraphics[width=0.5\textwidth]{fig/x-f5-baseline}}

Following Lindbo \& Tornberg (2011),%
\footnote{However, here $M$ is the number of subintervals and is
therefore even. (In the Lindbo \& Tornberg paper, the same letter
was used to denote number of grid points, which is odd.)}
we introduce the integer wavenumber $\kappa \in \{ -M/2, \ldots,
0, \ldots, M/2\}$ and the scaled wavenumber $k = 2\pi \kappa /
L$. The largest integer wavenumber is denoted by
$\kappa_\infty = M/2$, and the corresponding $k$ is $k_\infty =
2\pi (M/2)/L = \pi M / L$. In the first plot (Error vs $P$),
$\xi$ is selected as $\xi = \pi (M/L) / 12$, which means that
$k_\infty / \xi = 12$ in that plot. For smaller values of $P$,
this is to the right of the ``sweet spot'' in 2P (third plot) and
0P (fifth plot). For some reason the error increase with
$k_\infty/\xi$ is slower in 1P (fourth plot) than in 2P and 0P.

The reason that the error increases with $k_\infty/\xi$ when
nonperiodic directions are present is explained in af Klinteberg,
Saffar Shamshirgar, Tornberg (2017), section~5.3. For the
Gaussian window function, this can be understood as the Gaussian
$e^{-\xi^2r^2/(1-\eta)}$ needing to be properly truncated when
$\eta < 1$ (which is always the case in the first plot).%
\footnote{The reason the error grows with $k_\infty/\xi$ is
probably that the extension that we do, i.e.\ adding $hP$ to the
side length and $P$ to the grid size, becomes more and more
insufficient as $h$ decreases.}

\section{The 0P fix}

In af Klinteberg, Saffar Shamshirgar, Tornberg (2017), the
suggested fix is to extend the box by
\[
  \delta_L = \begin{cases}
    hP & \text{if $\eta \geq 1$}, \\
    \max(hP, \sqrt{2(1-\eta)m^2/\xi^2}) & \text{if $\eta < 1$},
  \end{cases}
\]
where $m = C \sqrt{\pi P}$ (where $C=0.976$ in the 0P Stokes
paper), i.e.\ $\tilde{L} = L + \delta_L$ is the side length of
the extended box. This strategy is used for the Gaussian window
in 0P electrostatics in the current code. For the Kaiser windows,
we instead use
\[
  \delta_L = \max(hP, \sqrt{8 (\beta/P) /\xi^2}),
\]
and since $\beta/P = 2.5$, this is actually $\delta_L = \max(hP,
\sqrt{20}/\xi)$.

As can be seen in the plots below, these selections completely
flatten the error curves (in plots 2--5) so that the error no
longer grows with $k_\infty/\xi$. Of course, $\delta_L$ is
rounded up so that $\delta_M = \delta_L/h$ becomes an (even)
integer, where $\tilde{M} = M + \delta_M$. We introduce
\begin{align*}
  \delta_{M,\text{rem,Gauss}} &= 2 \times
  \text{ceil}(\sqrt{2(1-\eta)m^2/\xi^2}/(2h)), \\
  \delta_{M,\text{rem,Kaiser}} &= 2 \times
  \text{ceil}(\sqrt{8 (\beta/P) /\xi^2}/(2h)).
\end{align*}

{\noindent\centering
\includegraphics[width=0.5\textwidth]{fig/x-f1-0Psol}\\%
\includegraphics[width=0.5\textwidth]{fig/x-f2-0Psol}%
\includegraphics[width=0.5\textwidth]{fig/x-f3-0Psol}\\%
\includegraphics[width=0.5\textwidth]{fig/x-f4-0Psol}%
\includegraphics[width=0.5\textwidth]{fig/x-f5-0Psol}}

The following table shows $\delta_{M,\text{rem,Gauss}}$ and
$\delta_{M,\text{rem,Kaiser}}$ as functions of $P$, for the
setting in the first plot (Error vs $P$), where $M=28$. This data is
the same for all periodicities (but not applicable to 3P of
course).
\begin{center}
  \begin{tabular}{ccc}
    \toprule
    $P$ & $\delta_{M,\text{rem,Gauss}}$ & $\delta_{M,\text{rem,Kaiser}}$ \\
    \midrule
     2 & 14 & 18 \\
     4 & 18 & 18 \\
     6 & 22 & 18 \\
     8 & 24 & 18 \\
    10 & 26 & 18 \\
    12 & 28 & 18 \\
    14 & 28 & 18 \\
    16 & 30 & 18 \\
    18 & 30 & 18 \\
    20 & 30 & 18 \\
    22 & 30 & 18 \\
    24 & 30 & 18 \\
    26 & 28 & 18 \\
    28 & 28 & 18 \\
    \bottomrule
  \end{tabular}
\end{center}
Note that $\delta_M = \max(P, \delta_{M,\text{rem}})$ for each
window function. This means that the actual choices are as
follows.
\begin{center}
  \begin{tabular}{ccccccc}
    \toprule
    $P$ & $\delta_M$ (Gauss) & $\delta_M$ (Kaiser) & $\tilde{M}$
    (Gauss) & $\tilde{M}$ (Kaiser) & $\tilde{M}/M$ (Gauss) &
    $\tilde{M}/M$ (Kaiser) \\
    \midrule
     2 & 14 & 18 & 42 & 46 & 1.5000 & 1.6429 \\
     4 & 18 & 18 & 46 & 46 & 1.6429 & 1.6429 \\
     6 & 22 & 18 & 50 & 46 & 1.7857 & 1.6429 \\
     8 & 24 & 18 & 52 & 46 & 1.8571 & 1.6429 \\
    10 & 26 & 18 & 54 & 46 & 1.9286 & 1.6429 \\
    12 & 28 & 18 & 56 & 46 & 2.0000 & 1.6429 \\
    14 & 28 & 18 & 56 & 46 & 2.0000 & 1.6429 \\
    16 & 30 & 18 & 58 & 46 & 2.0714 & 1.6429 \\
    18 & 30 & 18 & 58 & 46 & 2.0714 & 1.6429 \\
    20 & 30 & 20 & 58 & 48 & 2.0714 & 1.7143 \\
    22 & 30 & 22 & 58 & 50 & 2.0714 & 1.7857 \\
    24 & 30 & 24 & 58 & 52 & 2.0714 & 1.8571 \\
    26 & 28 & 26 & 56 & 54 & 2.0000 & 1.9286 \\
    28 & 28 & 28 & 56 & 56 & 2.0000 & 2.0000 \\
    \bottomrule
  \end{tabular}
\end{center}
Note that neither $\delta_{M,\text{rem,Gauss}}$ nor
$\delta_{M,\text{rem,Kaiser}}$ changes when $M/L$ changes
(assuming $\xi \sim M/L$); they are functions only of $P$
($\delta_{M,\text{rem,Kaiser}}$ is even constant). This means
that $\delta_M$ itself is a function only of $P$, not of $M$ or
anything else.

\section{The 2P fix}

The 2P fix is to always increase $M$ by $P+6$ in the nonperiodic
directions, which corresponds to $\delta_M = P+6$. This does not
seem completely unreasonable in principle given that we observed
above that $\delta_M$ is a function only of $P$ in the 0P fix.
The result is shown in the plots below. In the 1P case, this
seems to be enough to completely flatten the curves, while in 2P
and 0P it flattens the curves somewhat -- enough to make the
first plot look good (at $k_\infty/\xi=12$).

{\noindent\centering
\includegraphics[width=0.5\textwidth]{fig/x-f1-2Psol}\\%
\includegraphics[width=0.5\textwidth]{fig/x-f2-2Psol}%
\includegraphics[width=0.5\textwidth]{fig/x-f3-2Psol}\\%
\includegraphics[width=0.5\textwidth]{fig/x-f4-2Psol}%
\includegraphics[width=0.5\textwidth]{fig/x-f5-2Psol}}

With $\delta_M=P+6$, the choices look like this, independent of
window function.
\begin{center}
  \begin{tabular}{cccc}
    \toprule
    $P$ & $\delta_M$ & $\tilde{M}$ & $\tilde{M}/M$ \\
    \midrule
     2 &  8 & 36 & 1.2857 \\
     4 & 10 & 38 & 1.3571 \\
     6 & 12 & 40 & 1.4286 \\
     8 & 14 & 42 & 1.5000 \\
    10 & 16 & 44 & 1.5714 \\
    12 & 18 & 46 & 1.6429 \\
    14 & 20 & 48 & 1.7143 \\
    16 & 22 & 50 & 1.7857 \\
    18 & 24 & 52 & 1.8571 \\
    20 & 26 & 54 & 1.9286 \\
    22 & 28 & 56 & 2.0000 \\
    24 & 30 & 58 & 2.0714 \\
    26 & 32 & 60 & 2.1429 \\
    28 & 34 & 62 & 2.2143 \\
    \bottomrule
  \end{tabular}
\end{center}
In this situation, the 2P fix results in a smaller $\delta_M$
than the 0P fix for $P < 24$ for the Gaussian window, and for $P
< 12$ for the Kaiser window. Since both fixes seem to be working,
one could simply pick the one with the smallest $\delta_M$ for a
given $P$.

\section{Variations}

As the plots below show, the 2P fix ($\delta_M = P+6$) seems to
work for different original grid sizes (here tested with $M = 20,
24, 28, 32$), i.e.\ there is no strong dependence on $M$.

{\noindent\centering
\includegraphics[width=0.5\textwidth]{fig/x-f1-var1-M20}%
\includegraphics[width=0.5\textwidth]{fig/x-f1-var1-M24}\\%
\includegraphics[width=0.5\textwidth]{fig/x-f1-var1-M28}%
\includegraphics[width=0.5\textwidth]{fig/x-f1-var1-M32}}

Next, we try different variants on the 2P fix, namely $\delta_M =
P+2, P+4, P+6$ and $P+8$. It seems one could get away with $P+4$,
but $P+6$ takes down the Kaiser 2P curve a bit more (compare for
example at $P=10$).

{\noindent\centering
\includegraphics[width=0.5\textwidth]{fig/x-f1-var2-plus2}%
\includegraphics[width=0.5\textwidth]{fig/x-f1-var2-plus4}\\%
\includegraphics[width=0.5\textwidth]{fig/x-f1-var2-plus6}%
\includegraphics[width=0.5\textwidth]{fig/x-f1-var2-plus8}}

While $P+6$ seems optimal for the 2P case, it seems $P+4$ is
enough for the 0P case, and in the 1P case it is sufficient to
add only $P$.

\section{The more complicated relation}

Above, we used the simple relation
\begin{equation}
  \label{eq:simple-xi}
  \xi = \frac{\pi}{12}\frac{M}{L},
\end{equation}
which is probably what we want to use when producing the
Error-vs-$P$ plots. However, in other circumstances, we like to
use the more complicated relation
\begin{equation}
  \label{eq:complicated-M}
  \frac{M}{L} = \frac{\sqrt{3}\xi}{\pi} \sqrt{W \left(
  \frac{4Q^{2/3}}{3L^2(\pi\xi\varepsilon_*^2)^{2/3}}
  \right)},
\end{equation}
where $W$ is the Lambert $W$ function and $\varepsilon_*$ is an
error tolerance (and $Q = \sum_{n=1}^N q_n^2$). This relation
can also be inverted (i.e.\ solved for $\xi$), which yields
\begin{equation}
  \label{eq:complicated-xi}
  \xi = \frac{\pi}{\sqrt{2}} \frac{M}{L}
  \frac{1}{\sqrt{W\left(\dfrac{4Q}{\pi^2 L^2 \varepsilon_*^2 M}\right)}}.
\end{equation}
We have established a relation between $P$ and the error
tolerance, which is given by the table below.
\begin{center}
  \begin{tabular}{ccc}
    \toprule
    $\varepsilon \geq \num{5e-4}$ &:& $P=4$ \\
    $\varepsilon \geq \num{5e-6}$ &:& $P=6$ \\
    $\varepsilon \geq \num{5e-8}$ &:& $P=8$ \\
    $\varepsilon \geq \num{1e-10}$ &:& $P=10$ \\
    $\varepsilon \geq \num{1e-12}$ &:& $P=12$ \\
    $\varepsilon \geq \num{1e-13}$ &:& $P=14$ \\
    $\varepsilon \geq \num{2e-14}$ &:& $P=16$ \\
    \bottomrule
  \end{tabular}
\end{center}
Here, $\varepsilon = 10 \varepsilon_*$. This relation was based
precisely on the Error-vs-$P$ plots above. (This holds for the
Kaiser windows, but we will use the same values for all windows.)

Below we present Error-vs-$P$ plots where we use
\eqref{eq:complicated-xi} rather than \eqref{eq:simple-xi}.

\begin{center}
  \textbf{Baseline case}\\
  \includegraphics[width=0.5\textwidth]{fig/x-f1-baseline-adv}
\end{center}
Comparing with the first plot on page~1, we see that the Kaiser
error curves are a bit less steep (i.e.\ the error does not decay
quite as fast), but the additional error coming from the free
directions also seems to be quite a lot smaller. The reason for
this latter phenomenon must be that we have changed
$k_\infty/\xi$, probably decreased it. Therefore we are less
likely to end up in the bad part of the error curves. Recall
that before, $k_\infty/\xi=12$ always. Now the value depends on $M$ as
well as other parameters such as $P$ (which sets
$\varepsilon_*$), according to \eqref{eq:complicated-xi}.
For $P=4$, we have $k_\infty/\xi=5.6979$; for $P=8$, we have
$k_\infty/\xi=8.2361$; for $P=12$ we have $k_\infty/\xi=10.4949$;
and for $P=16$ we have $k_\infty/\xi=11.2040$. These values seem
to be closer to the ``sweet spot''.

\begin{center}
  \textbf{The 0P fix}\\
  \includegraphics[width=0.5\textwidth]{fig/x-f1-0Psol-adv}
\end{center}
As expected, the 0P fix solves the small remaining problem (only
really visible for 2P). The table of $\delta_M$ for this case
($M=28$) is found below.
\begin{center}
  \begin{tabular}{ccccc}
    \toprule
    $P$ & $\delta_M$ (Gauss) & $\delta_M$ (Kaiser) & $\tilde{M}$
    (Gauss) & $\tilde{M}$ (Kaiser) \\
    \midrule
     2 &  6 & 10 & 34 & 38 \\
     4 &  8 & 10 & 36 & 38 \\
     6 & 10 & 12 & 38 & 40 \\
     8 & 14 & 12 & 42 & 40 \\
    10 & 18 & 14 & 46 & 42 \\
    12 & 22 & 16 & 50 & 44 \\
    14 & 24 & 16 & 52 & 44 \\
    16 & 26 & 16 & 54 & 44 \\
    18 & 28 & 18 & 56 & 46 \\
    20 & 28 & 20 & 56 & 48 \\
    22 & 28 & 22 & 56 & 50 \\
    24 & 28 & 24 & 56 & 52 \\
    26 & 28 & 26 & 56 & 54 \\
    28 & 26 & 28 & 54 & 56 \\
    \bottomrule
  \end{tabular}
\end{center}
This can be compared with the table on page~4; unsurprisingly,
$\delta_M$ is smaller here than on page~4 (or at least never
larger). Note that since $\xi$ is determined by
\eqref{eq:complicated-xi}, it is no longer the case that $\xi$ is
directly proportional to $M/L$; rather, $\xi = C(M) \times M/L$,
where the prefactor $C(M)$ is given by
\[
  C(M) = \frac{\pi}{\sqrt{2}}
  \frac{1}{\sqrt{W\left(\dfrac{4Q}{\pi^2 L^2 \varepsilon_*^2 M}\right)}}.
\]
Here, $Q$ depends on the particle system, and $L$ on the box, so
the prefactor depends on the problem. Furthermore,
$\varepsilon_*^2$ is connected to $P$, so the prefactor depends
on $P$. Lastly, the prefactor contains $M$, so even for the same
problem and tolerance, the prefactor and thus $\delta_M$ may
change when varying $M$. This seems to be very far from a simple
rule such as $P+6$. Note in particular that as $M \to \infty$,
the value of the $W$ function approaches zero (since $W(0)=0$),
and thus $C(M) \to \infty$. This in turn means that $\xi \to
\infty$, and this is true even if $M/L$ is fixed. As $\xi \to
\infty$, we will have $\delta_{M,\text{rem}} \to 0$. This is in
fact good news, since it means that $\delta_M$ will simply be
equal to $P$ in this situation. It would be far worse if
$\delta_{M,\text{rem}}$ started to grow, but that can only happen
if $\xi L/M \to 0$, which is equivalent to $C(M) \to 0$, which is
in turn equivalent to
\[
  W\left(\frac{4Q}{\pi^2 L^2 \varepsilon_*^2 M}\right) \to \infty.
\]
The $W$ function is strictly increasing for nonnegative
arguments, so the way to make this happen is if
\[
  \frac{4Q}{\pi^2 L^2 \varepsilon_*^2 M} \to \infty.
\]
But this cannot happen for a fixed particle system and tolerance,
since $M$ cannot be selected arbitrarily small (it cannot be
smaller than 2, and even that would be silly). Thus, $\delta_M$
cannot be larger than the value it takes for the smallest $M$ one
considers (for fixed $Q$, $L$, $P$ and $\varepsilon_*$).

This leads us to the question: what \emph{is} the smallest $M$
that we can consider? Our main restriction is that $r_c \leq 1$,
and the relation between $r_c$ and $\xi$ is
\[
  \xi = \frac{1}{r_c} \sqrt{W \left( \frac{1}{\varepsilon_*}
  \sqrt{\frac{Q}{2L^3}} \right)},
\]
which means that the restriction is simply
\[
  \xi \geq \sqrt{W \left( \frac{1}{\varepsilon_*}
  \sqrt{\frac{Q}{2L^3}} \right)}.
\]
The number in the right-hand side depends on the particle system
(through $Q$ and $L$) and the selected tolerance (through
$\varepsilon_*$). For the test system that we use to compute the
plots in this document, we have $Q=31.4433$ and $L=1$. The
tolerance is related to $P$ through the table on page~8. The
lower bound for $\xi$ in this case is shown below.
\begin{center}
  \begin{tabular}{cc}
    \toprule
    $P$ & Lower bound for $\xi$ \\
    \midrule
    2 & 3.01255 \\
    4 & 3.01255 \\
    6 & 3.64671 \\
    8 & 4.19788 \\
    10 & 4.85252 \\
    12 & 5.28958 \\
    14 & 5.49597 \\
    16 & 5.63603 \\
    18 & 5.94532 \\
    20 & 5.94532 \\
    22 & 5.94532 \\
    24 & 5.94532 \\
    26 & 5.94532 \\
    28 & 5.94532 \\
    \bottomrule
  \end{tabular}
\end{center}
Plugging this into \eqref{eq:complicated-M}, we get $M$ at the
bounding value (first unrounded, then rounded):
\begin{center}
  \begin{tabular}{ccc}
    \toprule
    $P$ & $M$ for $\xi$ as in the table above & Rounded up to even number \\
    \midrule
    2 & 5.71089 & 6 \\
    4 & 5.71089 & 6 \\
    6 & 8.39782 & 10 \\
    8 & 11.1495 & 12 \\
    10 & 14.9207 & 16 \\
    12 & 17.7423 & 18 \\
    14 & 19.1593 & 20 \\
    16 & 20.1518 & 22 \\
    18 & 22.432 & 24 \\
    20 & 22.432 & 24 \\
    22 & 22.432 & 24 \\
    24 & 22.432 & 24 \\
    26 & 22.432 & 24 \\
    28 & 22.432 & 24 \\
    \bottomrule
  \end{tabular}
\end{center}
I don't know if this is the smallest $M$ (if $\xi$ increases in
\eqref{eq:complicated-M}, are we sure that $M$ also increases?),
but I would guess this is pretty much the smallest. The
corresponding values for $\delta_M$ are shown in the table below.
\begin{center}
  \begin{tabular}{cccccc}
    \toprule
    $P$ & $M$ & $\delta_M$ (Gauss) & $\delta_M$ (Kaiser) & $\tilde{M}$
    (Gauss) & $\tilde{M}$ (Kaiser) \\
    \midrule
     2 &  6 &  6 & 10 & 12 & 16 \\
     4 &  6 &  8 & 10 & 14 & 16 \\
     6 & 10 & 12 & 12 & 22 & 22 \\
     8 & 12 & 14 & 12 & 26 & 24 \\
    10 & 16 & 20 & 14 & 36 & 30 \\
    12 & 18 & 22 & 16 & 40 & 34 \\
    14 & 20 & 24 & 16 & 44 & 36 \\
    16 & 22 & 26 & 16 & 48 & 38 \\
    18 & 24 & 28 & 18 & 52 & 42 \\
    20 & 24 & 28 & 20 & 52 & 44 \\
    22 & 24 & 28 & 22 & 52 & 46 \\
    24 & 24 & 28 & 24 & 52 & 48 \\
    26 & 24 & 28 & 26 & 52 & 50 \\
    28 & 24 & 28 & 28 & 52 & 52 \\
    \bottomrule
  \end{tabular}
\end{center}
These should be the largest values of $\delta_M$, and therefore
also the largest ratios $\tilde{M}/M$. It seems the Error-vs-$P$
plot still looks okay with these choices (although $M=6$ is a
silly choice of course). Finally, I check the Error-vs-$P$ plot
for $M=32$, and it looks like this:\\[\baselineskip]
\begin{minipage}{0.5\textwidth}\centering
  \textbf{The 0P fix, M=32}\\
  \includegraphics[width=\textwidth]{fig/x-f1-0Psol-adv-M32}
\end{minipage}%
\begin{minipage}{0.5\textwidth}\centering
  \textbf{The 0P fix, M=36}\\
  \includegraphics[width=\textwidth]{fig/x-f1-0Psol-adv-M36}
\end{minipage}
Not sure why the 2P Kaiser curve behaves strange for $P$ between
16 and 22. Whatever happens seems to become even worse for
$M=36$. Maybe $\delta_M$ becomes too small somehow? I notice that
for $M=40$, a bug prevents the 1P code from running (this could
be in the direct summation though). The other codes can run, and
the result is shown below:
\begin{center}
  \textbf{The 0P fix, M=40}\\
  \includegraphics[width=0.5\textwidth]{fig/x-f1-0Psol-adv-M40}
\end{center}
Strange, the Kaiser 2P curve flattens out completely and then
joins the Gaussian curves. And $\delta_M$ is more or less the
same as its maximum value (table on page~12), just the maximum
minus 2 in some cases. I don't really see how this would explain
the problem, but it is still possible that $\delta_M$ is too
small.

Anyway, time to move on to the 2P fix, i.e.\ $\delta_M = P + k$
for some constant $k$ (we try $k=2, 4, 6, 8$). We will try this
fix for $M=16$, $M=28$, $M=32$ and possibly some larger values of
$M$ as well. First of all, we note that the baseline case itself
becomes worse for $M=16$ than for $M=28$:\\[\baselineskip]
\begin{minipage}{0.5\textwidth}\centering
  \textbf{Baseline case, M=16}\\
  \includegraphics[width=\textwidth]{fig/x-f1-var3-plus0-M16}
\end{minipage}%
\begin{minipage}{0.5\textwidth}\centering
  \textbf{Baseline case, M=28}\\
  \includegraphics[width=\textwidth]{fig/x-f1-baseline-adv}
\end{minipage}
One can't help but wonder what happens to the baseline case for
$M=36$ and $M=40$:\\[\baselineskip]
\begin{minipage}{0.5\textwidth}\centering
  \textbf{Baseline case, M=36}\\
  \includegraphics[width=\textwidth]{fig/x-f1-var3-plus0-M36}
\end{minipage}%
\begin{minipage}{0.5\textwidth}\centering
  \textbf{Baseline case, M=40}\\
  \includegraphics[width=\textwidth]{fig/x-f1-var3-plus0-M40}
\end{minipage}
Something strange clearly happens to the 2P case. It could be
that some other unrelated error also appears, which cannot be
decreased by increasing $\delta_M$?

Moving on, it's time to check $\delta_M = P+k$ for $M=16$.\\[\baselineskip]
\begin{minipage}{0.5\textwidth}\centering
  \textbf{P+2, M=16}\\
  \includegraphics[width=\textwidth]{fig/x-f1-var3-plus2-M16}
\end{minipage}%
\begin{minipage}{0.5\textwidth}\centering
  \textbf{P+4, M=16}\\
  \includegraphics[width=\textwidth]{fig/x-f1-var3-plus4-M16}
\end{minipage}\\
\begin{minipage}{0.5\textwidth}\centering
  \textbf{P+6, M=16}\\
  \includegraphics[width=\textwidth]{fig/x-f1-var3-plus6-M16}
\end{minipage}%
\begin{minipage}{0.5\textwidth}\centering
  \textbf{P+8, M=16}\\
  \includegraphics[width=\textwidth]{fig/x-f1-var3-plus8-M16}
\end{minipage}
We see that for $M=16$, it is enough to choose $\delta_M=P+2$.

Now for $M=28$:\\[\baselineskip]
\begin{minipage}{0.5\textwidth}\centering
  \textbf{P+2, M=28}\\
  \includegraphics[width=\textwidth]{fig/x-f1-var3-plus2-M28}
\end{minipage}%
\begin{minipage}{0.5\textwidth}\centering
  \textbf{P+4, M=28}\\
  \includegraphics[width=\textwidth]{fig/x-f1-var3-plus4-M28}
\end{minipage}\\
\begin{minipage}{0.5\textwidth}\centering
  \textbf{P+6, M=28}\\
  \includegraphics[width=\textwidth]{fig/x-f1-var3-plus6-M28}
\end{minipage}%
\begin{minipage}{0.5\textwidth}\centering
  \textbf{P+8, M=28}\\
  \includegraphics[width=\textwidth]{fig/x-f1-var3-plus8-M28}
\end{minipage}
It seems $P+2$ is enough here as well.

Next is $M=32$:\\[\baselineskip]
\begin{minipage}{0.5\textwidth}\centering
  \textbf{P+2, M=32}\\
  \includegraphics[width=\textwidth]{fig/x-f1-var3-plus2-M32}
\end{minipage}%
\begin{minipage}{0.5\textwidth}\centering
  \textbf{P+4, M=32}\\
  \includegraphics[width=\textwidth]{fig/x-f1-var3-plus4-M32}
\end{minipage}\\
\begin{minipage}{0.5\textwidth}\centering
  \textbf{P+6, M=32}\\
  \includegraphics[width=\textwidth]{fig/x-f1-var3-plus6-M32}
\end{minipage}%
\begin{minipage}{0.5\textwidth}\centering
  \textbf{P+8, M=32}\\
  \includegraphics[width=\textwidth]{fig/x-f1-var3-plus8-M32}
\end{minipage}
For $M=32$ it seems $P+4$ is the best choice; $P+2$ isn't quite
enough.

Finally we try $M=40$:\\[\baselineskip]
\begin{minipage}{0.5\textwidth}\centering
  \textbf{P+2, M=40}\\
  \includegraphics[width=\textwidth]{fig/x-f1-var3-plus2-M40}
\end{minipage}%
\begin{minipage}{0.5\textwidth}\centering
  \textbf{P+4, M=40}\\
  \includegraphics[width=\textwidth]{fig/x-f1-var3-plus4-M40}
\end{minipage}\\
\begin{minipage}{0.5\textwidth}\centering
  \textbf{P+6, M=40}\\
  \includegraphics[width=\textwidth]{fig/x-f1-var3-plus6-M40}
\end{minipage}%
\begin{minipage}{0.5\textwidth}\centering
  \textbf{P+8, M=40}\\
  \includegraphics[width=\textwidth]{fig/x-f1-var3-plus8-M40}
\end{minipage}
Here it seems one has to go up to $P+8$. On the other hand, this
fixes the problem better than the 0P fix.

I will try to run with $M=64$ as well, since it seems a bit like
the problem becomes worse and worse as $M$ grows. If that's the
case, one might be worried that not even $P+8$ would be enough
for large enough $M$. Results (1P cannot be ran for $M=64$
either due to the bug):
\begin{center}
  \textbf{Baseline case, M=64}\\
  \includegraphics[width=0.5\textwidth]{fig/x-f1-var3-plus0-M64}
\end{center}
Note that both window functions are affected by the bad
behaviour. However, it only happens in the 2P case (but of
course, we cannot see the 1P case here).\\[\baselineskip]
\begin{minipage}{0.5\textwidth}\centering
  \textbf{P+2, M=64}\\
  \includegraphics[width=\textwidth]{fig/x-f1-var3-plus2-M64}
\end{minipage}%
\begin{minipage}{0.5\textwidth}\centering
  \textbf{P+4, M=64}\\
  \includegraphics[width=\textwidth]{fig/x-f1-var3-plus4-M64}
\end{minipage}\\
\begin{minipage}{0.5\textwidth}\centering
  \textbf{P+6, M=64}\\
  \includegraphics[width=\textwidth]{fig/x-f1-var3-plus6-M64}
\end{minipage}%
\begin{minipage}{0.5\textwidth}\centering
  \textbf{P+8, M=64}\\
  \includegraphics[width=\textwidth]{fig/x-f1-var3-plus8-M64}
\end{minipage}
This seems to confirm the worries. We probably need to either
understand why this happens only for 2P and fix it, or adapt the
rule $P+k$ so that $k$ depends on $M$.

This section was about the more complicated relation between
$\xi$ and $M$, but the last results begs the question if $M=64$
looks bad also using the simpler relation \eqref{eq:simple-xi}.
Therefore we finish with a battery of such tests.
\begin{center}
  \textbf{Baseline case ($\xi=\pi(M/L)/12$), M=64}\\
  \includegraphics[width=0.5\textwidth]{fig/x-f1-var4-plus0-M64}
\end{center}
Interestingly, the 1P code worked in this case. So the error must
be related to the relation between $M$ and $\xi$. It seems the 2P
error was not very affected by changing relation, but for the 0P
code, the error went up. Now we can also see that the 1P code has
an error between the 2P and 0P codes.\\[\baselineskip]
\begin{minipage}{0.5\textwidth}\centering
  \textbf{($\xi=\pi(M/L)/12$) P+2, M=64}\\
  \includegraphics[width=\textwidth]{fig/x-f1-var4-plus2-M64}
\end{minipage}%
\begin{minipage}{0.5\textwidth}\centering
  \textbf{($\xi=\pi(M/L)/12$) P+4, M=64}\\
  \includegraphics[width=\textwidth]{fig/x-f1-var4-plus4-M64}
\end{minipage}\\
\begin{minipage}{0.5\textwidth}\centering
  \textbf{($\xi=\pi(M/L)/12$) P+6, M=64}\\
  \includegraphics[width=\textwidth]{fig/x-f1-var4-plus6-M64}
\end{minipage}%
\begin{minipage}{0.5\textwidth}\centering
  \textbf{($\xi=\pi(M/L)/12$) P+8, M=64}\\
  \includegraphics[width=\textwidth]{fig/x-f1-var4-plus8-M64}
\end{minipage}
Here the results seem rather similar to those shown on page~19.
So the problems with $M=64$ seems to hold for both
$M$-$\xi$-relations. Then it might be enough to do tests with the
simple relation; it seems this captures all problems already (we
just didn't try large enough $M$ before).

We will also try the 0P fix for $M=64$, both using the simple and
complicated relations between $M$ and $\xi$.\\[\baselineskip]
\begin{minipage}{0.5\textwidth}\centering
  \textbf{($\xi=\pi(M/L)/12$) 0P fix, M=64}\\
  \includegraphics[width=\textwidth]{fig/x-f1-var4-0Pfix-M64}
\end{minipage}%
\begin{minipage}{0.5\textwidth}\centering
  \textbf{(Complicated $M$-$\xi$-relation) 0P fix, M=64}\\
  \includegraphics[width=\textwidth]{fig/x-f1-var3-0Pfix-M64}
\end{minipage}
We see that the 0P fix is also not enough for $M=64$. This is
expected. It seems that $\delta_M$ really should depend on $M$
somehow, but it doesn't in the 0P fix (not in the right way at
least).

\section{Finding out how much is needed}

We return to the simple relation between $M$ and $\xi$, i.e.\
\eqref{eq:simple-xi}, and investigate how large $k$ in $\delta_M
= P+k$ needs to be to remove the bulge in the error curve, as a
function of $M$ and the periodicity.

\section{Timing tests}

TODO: These are done, and I will do no more timing. But write
down the conclusion here (which I think was to use $P+4$ or $P+6$
and then round up $M$ to the nearest multiple of 4).

([The only reason to complicate the simplest rule (which is to use
$\delta_M = P+6$ in all cases) would be if it makes a difference
to the runtime.])

\end{document}
